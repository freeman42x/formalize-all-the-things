### Formalization of Concepts

#### Intelligence

Let \( A \) be an agent and \( G \) be the set of goals \( A \) aims to achieve. Then the Intelligence \( I(A) \) can be defined as:

\[
I(A) = \frac{\text{Number of goals achieved by } A}{\text{Total number of goals in } G}
\]

Alternatively, if we use probability:

\[
I(A) = P(\text{"Achieving goals in set } G \text{" | Given } A \text{"})
\]

#### Ethics

Let \( S \) be a society, \( A_i \) be agents in that society, and \( W(A_i) \) be the well-being of agent \( A_i \). Then Ethics \( E(S) \) can be formalized as:

\[
E(S) = \max_{\text{goals}} \left( \sum_{i=1}^{n} W(A_i) \right)
\]

Where the maximization is over all possible combinations of individual agent goals such that the well-being of all agents in \( S \) is maximized.

### Implementation

For implementing intelligent agents that meet ethics constraints using compile-time proofs:

1. **Define Agent Goals**: Each agent \( A_i \) should have a formally defined set of goals \( G_i \).
2. **Define Well-being Function**: \( W(A_i) \) must be formally defined based on agent's goals and societal impact.
3. **Compile-time Proofs**: Implement a proof mechanism that proves, at compile-time, the goals \( G_i \) of each agent \( A_i \) align with \( E(S) \).

By ensuring that agent goals satisfy the maximum well-being constraint \( E(S) \) through compile-time proofs, we can have intelligent agents that are also ethical.